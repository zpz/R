\documentclass[12pt]{article}
\usepackage{techart}
\usepackage{ulem}

\begin{document}
\title{Spatial Statistics Resources in R}
\maketitle

This is an index of R packages that are mainly concerned about
spatial data and spatial analysis.
Listed information includes
scope, authors and maintainer,
history (launching time, latest version),
dependency on other spatially-concerned packages,
academic documentation (as opposed to technical documentation),
etc.

Projects that are apparently neither active nor depended upon by active
ones are not included.
Neither are those that are not likely to be useful to us.
Informally there are several broad groups:
(1) geostat and continuously-valued fields;
(2) spatial point patterns;
(3) connected to GIS, mapping, and geography.
Group~(1) is most relevant to us.
Group~(3) is least relevant to us at this time.

Overview references:
B.\ Ripley, R-News, 1(2), 2001;
CRAN Task View---Analysis of Spatial Data,
search for ``cran task view spatial'';
R Spatial Projects page at Spatial Analysis Laboratory of UIUC,
search for ``Rgeo'';
the R-sig-Geo mailing list,
search for ``r-sig-geo''.

\begin{itemize}
\item[ads]
Perform first- and second-order multi-scale analyses derived from
Ripley's K-function, for univariate, multivariate and marked mapped data
in rectangular, circular or irregular shaped sampling windows, with test
of statitical significance based on Monte Carlo simulations.

1.2-7, 2007-09-24.

By R.\ Pelissier and F.\ Goreaud.

\item[akima]
Linear or cubic spline interpolation for irregular gridded data.

0.2-3, 1998-08-20; 0.5-1, 2006-02-01.

Fortran code by H. Akima, R port by Albrecht Gebhardt.

\item[\emph{fields}]
Fields is for curve, surface and function fitting with an emphasis on
splines, spatial data and spatial statistics. The major methods include
cubic, robust, and thin plate splines, multivariate Kriging and Kriging
for large data sets. One main feature is any covariance function
implemented in R can be used for spatial prediction. There are also
useful functions for plotting and working with spatial data as images.
This package also contains an implementation of a sparse matrix methods
for large data sets.

1.0, 2001-07-27; 4.1, 2007-11-12.

By Doug Nychka.

\item[\emph{geoR}]
Geostatistical analysis including traditional, likelihood-based and
Bayesian methods.

Depends on \texttt{sp}.

1.0.0, 2001; 1.6.20, 2007.


By Paulo J.\ Ribeiro Jr.\ and Peter J.\ Diggle.

Paulo J.\ Ribeiro Jr.\  and Peter J.\ Diggle, R-News, 1(2):15--18, June 2001.

\item[\emph{geoRglm}]
Functions for inference in generalised linear spatial models. The
posterior and predictive inference is based on Markov chain Monte Carlo
methods. Package \texttt{geoRglm} is an extension to the package
\texttt{geoR}.

Depends on \texttt{geoR}.

0.4-1, 2002-02-13; 0.8-22, 2007-12-17.

By Ole F.\ Christensen and Paulo J.\ Ribeiro Jr.

Ole E.\ Christensen and Paulo J. Ribeiro Jr., R-News, 2(2), June 2002.

\item[? gstat]
variogram modelling; simple, ordinary and universal point or block
(co)kriging, sequential Gaussian or indicator (co)simulation; variogram
and variogram map plotting utility functions.

Depends on \texttt{sp}.

0.9-13, 2003-02-19; 0.9-43, 2008-02-03.

By Edzer J.\ Pebesma.

E.\ Pebesma, Computers \& Geosciences, 30:683--691, 2004.

\item[maps]
Display of maps. Projection code and larger maps are in separate
packages (mapproj and mapdata).

2.0-3, 2003-10-07; 2.0-39, 2007-10-14.

\item[maptools]
Set of tools for manipulating and reading geographic data, in particular
ESRI shapefiles; C code used from shapelib. It includes binary access to
GSHHS 1.5 shoreline files. The package also provides interface wrappers
for exchanging spatial objects with packages such as PBSmapping,
spatstat, maps, RArcInfo, Stata tmap, WinBUGS, Mondrian, and others.

Depends on \texttt{sp}.

0.2-2, 2003-08-12; 0.7-7, 2008-02-18.

By Nicholas Lewin-Koh and Roger Bivand.

\item[? ramps]
Bayesian geostatistical modeling of Gaussian processes using a
reparameterized and marginalized posterior sampling (RAMPS) algorithm
designed to lower autocorrelation in MCMC samples. Package performance
is tuned for large spatial datasets.

Depends on \texttt{fields}, \texttt{maps}.

0.6-1, 2007-12-16.

By Brian J. Smith, Jun Yan, and Mary Kathryn Cowles.

\item[\emph{RandomFields}]
Simulation of Gaussian and extreme value random fields; conditional
simulation; kriging.

1.0.0, 2001-06-01; 1.3.30, 2008-02-16.

By Martin Schlather.

Martin Schlather, R-News, 1(2):18--20, June 2001.

\item[RColorBrewer]
The packages provides palettes for drawing nice maps shaded according to
a variable.

1.0-2, 2007-10-21.

By Erich Neuwirth.

\item[regress]
Functions to fit Gaussian linear model by maximising the residual log
likelihood where the covariance structure can be written as a linear
combination of known matrices. Can be used for multivariate models and
random effects models. Easy straight forward manner to specify random
effects models, including random interactions.

0.1, 2004-06-15; 1.0-0, 2006-06-09.

By David Clifford and Peter McCullagh.

\item[rgdal]
Provides bindings to Frank Warmerdam's Geospatial Data Abstraction
Library (GDAL) ($>=$ 1.3.1) and access to projection/transformation
operations from the PROJ.4 library.
Links R to GIS.
A translator library for 60 raster geospatial data formats.

Depends on \texttt{sp}.

0.2-5, 2003-11-26; 0.5-24, 2008-02-05.

By Timothy H.\ Keitt, Roger Bivand, Edzer Pebesma, Barry Rowlington.

\item[\emph{sp}]
A package that provides classes and methods for spatial data. The
classes document where the spatial location information resides, for 2D
or 3D data. Utility functions are provided, e.g. for plotting data as
maps, spatial selection, as well as methods for retrieving coordinates,
for subsetting, print, summary, etc.
The intention of this package is to provide some sort of common
infrastructure to spatial packages.

0.7-3, 2005-04-28; 0.9-23, 2008-02-18.

By Edzer J.\ Pebesma and Roger Bivand.

Edzer J.\ Pebesma and Roger S.\ Bivand, R-News, 5(2), Nov 2005.

\item[spatgraphs]
Graphs, graph visualization and graph based summaries to be used as a
tool in spatial point pattern analysis.

0.52, 2007-11-14.

\item[\emph{spatial}]
(part of \texttt{VR})
Trend surface analysis, kriging, and point processes.

6.1-4, 1999-08-16; 7.2-41, 2008-02-16.

By B.\ D.\ Ripley.

Venables and Ripley, MASS, 4th ed., Springer, 2002.

\item[? spatialCovariance]
Functions that compute the spatial covariance matrix for the matern and
power classes of spatial models, for data that arise on rectangular
units. This code can also be used for the change of support problem and
for spatial data that arise on irregularly shaped regions like counties
or zipcodes by laying a fine grid of rectangles and aggregating the
integrals in a form of Riemann integration.

0.1, 2004-04-27; 0.5, 2005-05-09.

By David Clifford.

\item[spatstat]
A package for analysing spatial data, mainly Spatial Point Patterns,
including multitype/marked points and spatial covariates, in any
two-dimensional spatial region.

1.0-1, 2002-01-21; 1.12-7, 2008-02-11.

By Adrian Baddeley and Rolf Turner.

A.\ Baddeley and R.\ Turner, Journal of Statistical Software,
12(6):1--42, 2005.

\item[? spBayes]
Fits Gaussian univariate and multivariate models with Markov chain Monte
Carlo.

0.0-1, 2006-10-01; 0.0-6, 2007-04-10.

By Andrew O.\ Finley, Sudipto Banerjee, Bradley P.\ Carlin.

\item[spdep]
Spatial dependence; weighting schemes; statistics and models.

Depends on \texttt{sp}, \texttt{spam}, \texttt{tripack}.

0.1-2, 2002-03-20; 0.4-17, 2008-02-18.

By Roger Bivand.

\item[spgrass6]
Interpreted interface between GRASS 6 geographical information system
and R, based on starting R from within the GRASS environment.

Depends on \texttt{sp}, \texttt{rgdal}.

0.5-2, 2008-02-03.

\item[splancs]
Spatial and space-time point pattern analysis functions.

Depends on \texttt{sp}.

2.01-2, 2000-11-05; 2.01-23, 2007-08-23.

By Barry Rowlingson and Peter Diggle.

\item[spsurvey]
This group of functions implements algorithms required for design and
analysis of probability surveys such as those utilized by the U.S.
Environmental Protection Agency's Environmental Monitoring and
Assessment Program (EMAP).

Depends on \texttt{sp}.

1.6.2, 2007-03-07.

By Tom Kincaid and Tony Olsen.

\item[tgp]
Bayesian nonstationary, semiparametric nonlinear regression and design
by treed Gaussian processes with jumps to the limiting linear model
(LLM). Special cases also implemented include Bayesian linear models,
CART, treed linear models, stationary separable and isotropic Gaussian
processes. Provides 1-d and 2-d plotting functions (with projection and
slice capabilities) and tree drawing, designed for visualization of
tgp-class output. Sensitivity analysis and multi-resolution models are
supported, and a limited set of experimental design and adaptive
sampling functions are also provided, including ALM, ALC, and expected
improvement.

1.0-1, 2006-01-05; 2.0-4, 2008-01-23.

By Robert B.\ Gramacy and Matt A.\ Taddy.

\item[? tripack]
Triangulation of irregularly spaced data.

1.0-1, 1998-08-20; 1.2-11, 2007.

By R.\ J.\ Renka and Albrecht Gebhardt.

\item[vardiag]
Interactive variogram diagnostics.

0.1, 2003-05-25; 0.1-1, 2007-10-21.

By Ernst Glatzer.

\end{itemize}

\end{document}

